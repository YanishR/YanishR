%%%%%%%%%%%%%%%%%%%%%%%%%%%%%%%%%%%%%%%%%
% Medium Length Professional CV
% LaTeX Template
% Version 2.0 (8/5/13)
%
% This template has been downloaded from:
% http://www.LaTeXTemplates.com
%
% Original author:
% Trey Hunner (http://www.treyhunner.com/)
%
% Important note:
% This template requires the resume.cls file to be in the same directory as the
% .tex file. The resume.cls file provides the resume style used for structuring the
% document.
%
%%%%%%%%%%%%%%%%%%%%%%%%%%%%%%%%%%%%%%%%%

%----------------------------------------------------------------------------------------
%	PACKAGES AND OTHER DOCUMENT CONFIGURATIONS
%----------------------------------------------------------------------------------------

\documentclass{resume} % Use the custom resume.cls style

\usepackage[left=0.2in,top=0.2in,right=0.2in,bottom=0.15in]{geometry} % Document margins
\usepackage{soul}
\usepackage{fontawesome5}
\usepackage{hyperref}
\hypersetup{
    colorlinks=true,
    linkcolor=blue,
    filecolor=blue,
    urlcolor=blue,
    citecolor=blue
}


\name{Yanish Rambocus} % Your name
 % Your address

\capsdef{////}{\scshape}{1.3pt}{5pt}{1pt}
\address{\caps{55 Charteris Road, London, N4 3AA}}
\resetso
\sodef\an{}{.14mm}{1.35mm}{15mm}
\address{\an{+44 7951 255607 $|$ yanishrambocus@gmail.com}} % Your phone number and email


\begin{document}
%----------------------------------------------------------------------------------------
%	TECHNICAL STRENGTHS SECTION
%----------------------------------------------------------------------------------------

\vspace{-2.1em}
\begin{rSection}{Profile}   
    Software Engineer for three years developing business systems deployed to the Cloud,  mainly in Kotlin 
    and Terraform. Working for companies of different sizes and stages, I have led software projects, 
    prolifically contributed to codebases and mentored Junior developers, improving my technical 
    and interpersonal skills.\\
    Seeking the right opportunity to positively contribute while growing as an Engineer.
    \capsdef{////}{\scshape}{0.1pt}{2.5pt}{1pt}
    \vspace{3.1mm}\\
    \begin{tabular}{ @{\hspace{8mm}} l @{\hspace{13.5ex}} l }
        \vspace{1.1mm}
        \large{\caps{Languages}} & Kotlin, Python, Java\\
        \vspace{1.1mm}
        \large{\caps{Cloud Infrastructure}} & AWS, GCP, Terraform \\
        \vspace{1.1mm}
        \large{\caps{Databases}} & PostgreSQL, DynamoDB\\
        \vspace{1.1mm}
        \large{\caps{Messaging Queues}} & Kafka, SQS, Pub/Sub\\
        \vspace{1.1mm}
        \large{\caps{Other Technologies}} & AWS ECS \& Lambda, Kubernetes, Quarkus\\
        & Github Actions, CircleCI, Keycloak\\
    \end{tabular}

    \vspace{-1.5mm}
\end{rSection}

%----------------------------------------------------------------------------------------
%	WORK EXPERIENCE SECTION
%----------------------------------------------------------------------------------------

\begin{rSection}{Work Experience}
    \begin{rSubsection}{Kerno.io}{March 24 - Present}{Senior Software Engineer}{London, UK}
    \item Leveraged Github's APIs to appropriately project data for different types of users    
    \item Developed a K8s agent that aggregates events meaningfully for end users
    \item Implemented Github Actions to build, test PRs and push images on merges to \texttt{main}
  \end{rSubsection}
  \begin{rSubsection}{OVO Energy}{Dec 22 - Feb 24}{Software Engineer}{London, UK}
    \item Lead development of automation tools that transformed manual processes into auditable events
    \item Optimised CI/CD pipelines reducing average CI times by half and Lambda execution times by 12x 
    \item Managed team’s transition from CircleCI to Github Actions to narrow developer focus
    \item Developed Terraform code to deploy Cloud services, becoming proficient in Infrastructure-as-Code
  \end{rSubsection}
  \begin{rSubsection}{So Energy}{Dec 21 - Nov 22}{Junior Backend Engineer}{London, UK}
    \item Automated lead generation for customers interested in installing Solar Panels
    \item Developed features for payments and authorization services, maintained common libraries
    \item Managed microservices running on K8s engine hosted on GCP
  \end{rSubsection}

\end{rSection}
%----------------------------------------------------------------------------------------
%	EDUCATION SECTION
%----------------------------------------------------------------------------------------

\begin{rSection}{Education}
    \capsdef{////}{\scshape}{0.1pt}{2.5pt}{1pt}
    \begin{rSubsection}{University of Richmond}{May 2021}{BSc in Computer Science, \large{\caps{ GPA:}} 3.18}{Richmond, VA, USA}
    \item  Enrolled as a Science Scholar with a full tuition scholarship.
    \item \caps{Coursework: } Data and Discrete Structures, Computer Architecture, Algorithms,\\
        Design and implementation of Programming Languages, Computer Security, NLP, Artificial Intelligence.
    \item \caps{Mathematics Coursework: } Multivariate Calculus, Linear Algebra, Real Analysis.
    \item \caps{Business Coursework: } Introduction to Financial and Managerial Accounting, Business Statistics.
    \end{rSubsection}

\end{rSection}

\begin{rSection}{Coding Projects}
    \begin{rSubsection}{Natural Language Processing}{Fall 2020}{}{}
        \item Bot or Human tweet - Given a tweet, determine whether it emanates from human or bot.\\
           Leveraged Python's ML ecosytem to apply various machine learning techniques,
           querying Twitter's API for data.

    \end{rSubsection}

    \begin{rSubsection}{Artificial Intelligence}{Spring 2020}{}{}
        \item Chess - Calculate next best move using H-minimax. \\
            Developed a next-best-move program, as well as a Chess game to analyze the engine's performance.
        \item Sentiment Analysis - Determine a review's sentiment.\\
            Applied Python ML tools such as \texttt{sklearn} and \texttt{numpy} on IMDB datasets
    \end{rSubsection}

\end{rSection}

\vspace{1.4em}

\small  
\noindent
\faLinkedin \hspace{0.2em} \href{https://linkedin.com/in/yanish-rambocus-4a46b7155/}{\underline{Yanish Rambocus}}
\hspace{431pt}
\rlap{\faGithub \hspace{0.2em} \href{https://github.com/YanishR}{\underline{YanishR}}}

\end{document}
