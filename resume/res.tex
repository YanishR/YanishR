%%%%%%%%%%%%%%%%%%%%%%%%%%%%%%%%%%%%%%%%%
% Medium Length Professional CV
% LaTeX Template
% Version 2.0 (8/5/13)
%
% This template has been downloaded from:
% http://www.LaTeXTemplates.com
%
% Original author:
% Trey Hunner (http://www.treyhunner.com/)
%
% Important note:
% This template requires the resume.cls file to be in the same directory as the
% .tex file. The resume.cls file provides the resume style used for structuring the
% document.
%
%%%%%%%%%%%%%%%%%%%%%%%%%%%%%%%%%%%%%%%%%

%----------------------------------------------------------------------------------------
%	PACKAGES AND OTHER DOCUMENT CONFIGURATIONS
%----------------------------------------------------------------------------------------

\documentclass{resume} % Use the custom resume.cls style

\usepackage[left=0.2in,top=0.2in,right=0.2in,bottom=0.25in]{geometry} % Document margins
\usepackage{soul}
\usepackage{fontawesome5}
\usepackage{hyperref}
\hypersetup{
    colorlinks=true,
    linkcolor=blue,
    filecolor=blue,
    urlcolor=blue,
    citecolor=blue
}


\name{Yanish Rambocus} % Your name
 % Your address

\capsdef{////}{\scshape}{1.3pt}{5pt}{1pt}
\address{\caps{55 Charteris Road, London, N4 3AA}}
\resetso
\sodef\an{}{.14mm}{1.35mm}{15mm}
\address{\an{+44 7951 255607 $|$ yanishrambocus@gmail.com}} % Your phone number and email


\begin{document}
%----------------------------------------------------------------------------------------
%	TECHNICAL STRENGTHS SECTION
%----------------------------------------------------------------------------------------

\vspace{-2.1em}
\begin{rSection}{Profile}   
    Software Engineer for three years developing business applications deployed to the Cloud,  mainly in Kotlin 
    and Terraform. Working for companies of different sizes and stages, I have led software projects, 
    prolifically contributed to codebases and mentored Junior developers, improving my technical 
    and interpersonal skills. I am always keen to learn new technologies.
    Seeking the right opportunity to positively contribute while growing as an Engineer.\\
    \capsdef{////}{\scshape}{0.1pt}{2.5pt}{1pt}
    \vspace{1.5mm}\\
    %\begin{tabular}{ @{\hspace{3mm}} >{\bfseries}l @{\hspace{13.5ex}} l }
    \begin{tabular}{ @{\hspace{3mm}} l @{\hspace{13.5ex}} l }
        \vspace{.5mm}
        \large{\caps{Languages}} & Kotlin, Python, Java\\
        \vspace{.5mm}
        \large{\caps{Cloud Infrastructure}} & AWS, GCP, Terraform \\
        \vspace{.5mm}
        \large{\caps{Databases}} & PostgreSQL, DynamoDB\\
        \vspace{.5mm}
        \large{\caps{Messaging Queues}} & Kafka, SQS, Pub/Sub\\
        \vspace{.5mm}
        \large{\caps{Other Technologies}} & AWS ECS \& Lambda, Kubernetes, Quarkus\\
        & Github Actions, CircleCI, Keycloak\\
    \end{tabular}

    \vspace{-1.5mm}
\end{rSection}

%----------------------------------------------------------------------------------------
%	WORK EXPERIENCE SECTION
%----------------------------------------------------------------------------------------

\begin{rSection}{Work Experience}
    \begin{rSubsection}{Kerno.io}{Feb 24 - Present}{Senior Software Engineer}{London, UK}
    \item Leveraged Github's APIs to appropriately project data for different types of users    
    \item Developed a K8s agent that aggregates events meaningfully for end users
    \item Implemented Github Actions to build, test PRs and push images on merges to \texttt{main}
  \end{rSubsection}
  \begin{rSubsection}{OVO Energy}{Dec 22 - Feb 24}{Software Engineer}{London, UK}
    \item Lead development of automation tools that transformed manual processes into auditable events
    \item Optimised CI/CD pipelines reducing average CI times by half and Lambda execution times by 12x 
    \item Managed team’s transition from CircleCI to Github Actions to narrow developer focus
    \item Developed Terraform code to deploy Cloud services, becoming proficient in Infrastructure-as-Code
  \end{rSubsection}
  \begin{rSubsection}{So Energy}{Dec 21 - Nov 22}{Junior Backend Engineer}{London, UK}
    \item Programmed in Kotlin to develop a core ecosystem of microservices
    \item Worked in payments, authorization and common libraries, amongst others
    \item Managed docker containers on a K8s engine hosted on GCP
  \end{rSubsection}
  
%  \begin{rSubsection}{SpiRoL}{May - August 2019}{Development Researcher}{Richmond, VA, USA}
%    \item Collaborated to simulate the egress of groups within an environment when faced with a threat
%    \item Designed and implemented input system of application to ease customization of our simulation
%    \item Highly engaged with C++ and paradigms of the language, used XML as input format
%    \item Implemented triangulation for path planning of agents
%  \end{rSubsection}

\end{rSection}
%----------------------------------------------------------------------------------------
%	EDUCATION SECTION
%----------------------------------------------------------------------------------------

\begin{rSection}{Education}
    \capsdef{////}{\scshape}{0.1pt}{2.5pt}{1pt}
    \begin{rSubsection}{University of Richmond}{May 2021}{BSc in Computer Science, \large{\caps{ GPA:}} 3.18}{Richmond, VA, USA}
    \item  Enrolled as a Science Scholar with a full tuition scholarship.
    \item \caps{Coursework: } Data and Discrete Structures, Computer Architecture, Algorithms,\\
        Design and implementation of Programming Languages, Computer Security, NLP, Artificial Intelligence.
    \item \caps{Mathematics Coursework: } Multivariate Calculus, Linear Algebra, Real Analysis.
    \item \caps{Business Coursework: } Introduction to Financial and Managerial Accounting, Business Statistics.
    \end{rSubsection}

\end{rSection}

    

%\begin{rSubsection}{Photon Infotech, Inc}{June - August 2018}{Development Intern}{Deerfield, IL, USA}
 % \item Mentored by industry experts on creating a successful e-commerce platform 
  %\item Grasped and coded key tests to kickstart online store for the company
  %\item Used and worked extensively with commertools SDK and Java's Spring Framework
  %\item Familiarized myself with servers and back-end programming
%\end{rSubsection}

%------------------------------------------------


%\begin{rSubsection}{Schneider Electric}{January - March 2017}{Testing Intern}{Bangalore, India}
 % \vspace{-1mm}
  %\item Placed in Testing team to conduct unit tests
  %\item Conducted Test Driven Development
%\end{rSubsection}

%------------------------------------------------


\begin{rSection}{Coding Projects}
    \begin{rSubsection}{Natural Language Processing}{Fall 2020}{}{}
        \item Bot or Human tweet - Given a tweet, determine whether it emanates from human or bot.\\
           With Python's ML ecosytem, applied different machine learning techniques by 
           querying Twitter's API for data.

    \end{rSubsection}

    \begin{rSubsection}{Artificial Intelligence}{Spring 2020}{}{}
        \item Chess - Calculate next best move using H-minimax. \\
            Coded a chess game to analyze our program that calculates the next best move for a player.
        \item Sentiment Analysis - Determine a review's sentiment.\\
            Applied Python Machine Learning tools such as sklearn and numpy on IMDB  datasets
    \end{rSubsection}

    %\begin{rSubsection}{Expense Tracking Website}{June - August 2018}{Created web app to conveniently track expenses}{}
    %\item Used Node.js and Express.js to create RESTful APIs
    %\item Deployed a MongoDB database to access user information
    %\item Rendered static webpages with Jade template engine
  %\end{rSubsection}

  %\pagebreak
%\vspace{-1mm}
  %\begin{rSubsection}{Stock Trading Platform}{May 2017}{Created a web app to trade stocks}{}
    %\item Incorporated HTML, CSS, Python's Django and Flask frameworks and SQLite to create website to trade stocks.
    %\item Expanded knowledge of RESTful APIs, how to use them
    %\item Stocks and money could not be actually traded
  %\end{rSubsection}
\end{rSection}

\vspace{1em}

\small  
\noindent
\faLinkedin \hspace{0.2em} \href{https://linkedin.com/in/yanish-rambocus-4a46b7155/}{\underline{Yanish Rambocus}}
\hspace{431pt}
\rlap{\faGithub \hspace{0.2em} \href{https://github.com/YanishR}{\underline{YanishR}}}

%----------------------------------------------------------------------------------------
%	EXAMPLE SECTION
%----------------------------------------------------------------------------------------

%\begin{rSection}{Section Name}

%Section content\ldots

%\end{rSection}

%----------------------------------------------------------------------------------------

\end{document}
